\documentclass[11pt]{article}

  \usepackage[ngerman]{babel}
  \usepackage[T1]{fontenc}
  \usepackage[utf8]{inputenc} 
\usepackage[pdftex]{graphicx}
\usepackage{geometry,blindtext}
\geometry{a4paper,left=30mm,right=20mm, top=1cm, bottom=2cm, includeheadfoot}
\usepackage{float}

\usepackage{amsmath, amsthm, amssymb}

\usepackage[babel,german=quotes]{csquotes}

\usepackage{fancyhdr}
\pagestyle{fancy}
\fancyhf{}

\fancyhead[L]{\small{\textbf{Basistechniken Teilprüfung 2}}}

\fancyhead[C]{\small{David Brüggemann}}

\fancyhead[R]{\today}

\begin{document}


\begin{titlepage}
  \title{Dokumentation durchgeführter Arbeiten und Lerntagebuch}
  \author{David Brüggemann\\ Fachhochschule Südwestfalen}
  \date{\today}
\end{titlepage}

\maketitle
\fancyfoot[C]{}
\newpage

\tableofcontents
\fancyfoot[C]{}
\newpage

\section{Einleitung}


\section{Lernplanung}
Um den Überblick über das Semester und Prüfungen zu halten habe ich zunächst einen Semesterplan sowie einen Wochenplan und einen Prüfungslernplan erstellt.


\subsection{Wochenplan}
Um den Wochenplan zu erstellen habe ich meinen Kalender her genommen und die Vor und Nachbereitungen und ähnliche Wichtigkeiten in diesen bei GoogleCalender integriert. Aus diesem habe ich exemplarisch 2 Wochen entnommen um den Wochenplan dokumentieren zu können.


\begin{figure}[hbtp]
\centering
    \includegraphics[scale=0.6]{calendar_2012-06-04_2012-06-11.pdf}
    \caption{Auszug Wochenplan aus eigenem Kalender 4.6.-10.6.2012 mit Feiertag am 7.6.12}
    \end{figure}
    


\fancyfoot[C]{1}
\newpage
 
\begin{figure}[hbtp]
\centering
\includegraphics[scale=0.6]{calendar_2012-06-11_2012-06-18.pdf}
\caption{zweiter Auszug Wochenplan 11.6.-18.6.2012}
\end{figure}

\fancyfoot[C]{2}

\newpage


\subsection{Semesterplan}
Um alle wichtigen Termine, Abgaben, Tests und Prüfungen im Blick und Abrufbereit zu haben habe ich einen Semesterplan in Form einer Tabelle mit genau diesen Daten und Informationen erstellt. Diesen Plan habe ich in wichtigen Ordnern und Unterlagen für dieses Semester verteilt. 
\\Da es jede und oder jede zweite Woche eine Abgabe für Mathe 2, Programmieren 2 und Datenbanken gibt habe ich diese nicht mit in den Plan aufgenommen. Dazu kann man noch sagen, dass ich in der Regel gebe diese Aufgaben wenn möglich eine Woche vor Abgabepflicht ab damit ich zur größten Not noch 2 Termine, mit Nachfrist, zur Verfügung habe. 
\\\\
\begin{figure}
\begin{tabular}{}

\end{tabular}
\caption{Semesterplan in Tabellenform}
\end{figure}

\fancyfoot[C]{3}

\newpage

\subsection{Lernplan für Klausuren}
Nun noch den Klausurplan für dieses Semester. Hier beschreibe ich kurz, wieder in Tabellenform, wie viel Aufwand ich gedenke für die jeweiligen Klausuren einzuplanen. Zusätzlich in welchem Zeitraum hierfür gelernt wird in Form eines Bildes dargestellt. 
\\
Ich denke hier kann man auch dann schon Prioritäten setzen, sodass man für die Prüfungen sich Ziele setzen kann. 
\begin{figure}
\begin{tabular}{|l|l|}
\hline
Modul: & Zeitaufwand (in Stunden): \\
\end{tabular}
\caption{Der Zeitaufwände für das jeweilige Modul}
\end{figure}

Bild für Modulvorbereitung

\fancyfoot[C]{4}

\newpage


\section{Lerntagebuch}
Das Führen von Lerntagebüchern ist eine bewährte Methode, die eigene Lernpraxis zu dokumentieren, zu erkunden, zu überprüfen und möglicherweise zu verändern. \\
Für dieses Semester habe ich 4 Wochen ein Lerntagebuch geführt.

\subsection{Auszug Lerntagebuch für 2 Tage}
Exemplarisch gebe ich hier 2 Tage meines Lerntagebuches preis:Für den 22.05.2012 und den  \\
Ich habe mich für 2 Tage die mittig liegen da ich dort schon an das Schreiben gewöhnt hatte. An beiden Tagen lief das Lernen gut. An dem 22.05. habe ich mehr gemacht und am 31.05. hab ich mir mehr Freizeit genommen dennoch habe ich an beiden Tagen das gelernte verinnerlichen können.
\\
\begin{figure}
\begin{tabular}{|p{2.3cm}|p{4.5cm}|p{4.5cm}|p{3cm}|}
\end{tabular}
\caption{Lerntagebuch für den 22.05.2012 - Lernzeit insgesamt:  Minuten}
\end{figure}

\begin{figure}
\begin{tabular}{|p{2.3cm}|p{4.5cm}|p{4.5cm}|p{3cm}|}
\hline
Zeitdauer & Tätigkeit & Thema & Bewertung \\
\hline
\hline
08:00 - 09:30  
90 Minuten& Fragenstellen, Aufgabe abgeben & Datenbanken SQL & Abgenommen\\
\hline
09:30 - 11:00
90 Minuten & Bericht schreiben, besprechen & Basistechniken treffen für Miniprojekt & OK\\
\hline
11:30 - 12:30
60 Minuten & Wiederholen der Vorlesung & Mathe 2 - Determinanten & Zufriedenstellend\\
\hline
17:30 - 18:20
50 Minuten & lesen & GDI Bäume Text & OK\\
\hline

\end{tabular}
\caption{Lerntagebuch für den 31.05.2012 - Lernzeit insgesamt: 290 Minuten}
\end{figure}

\fancyfoot[C]{5}

\newpage

\subsection{Zusammenfassung und Reflexion Lerntagebuch}
Das Lerntagebuch ist an sich eine gute Sache. Ich konnte am Ende des Tages alles gelernte nochmal grob überblicken und konnte feststellen woran manche Einheiten gescheitert sind. Ich habe versucht darauf in der Tabelle aufmerksam zu machen indem ich in die Bewertungsspalte ein möglichst treffendes Wort dafür finde. \\ Wenn Lerneinheiten gescheitert sind lag es meistens daran, dass ich mir zu wenig Zeit dafür genommen habe oder ich nicht so wirklich Motiviert war.\\

\fancyfoot[C]{6}

\section{Auswertung Lerntagebuch/Workloaderfassung}


\end{document}
